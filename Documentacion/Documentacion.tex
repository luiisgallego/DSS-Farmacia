\documentclass[a4paper,12pt,oneside]{article}

\usepackage[spanish, es-tabla]{babel}
\usepackage[hidelinks]{hyperref}
\hypersetup{
    colorlinks=true,
    linkcolor=blue,
    filecolor=magenta,      
    urlcolor=cyan,
}
\usepackage{graphicx}
\usepackage{amssymb, amsmath, amsbsy}
\usepackage{mathptmx}	
\usepackage{float}
\usepackage{booktabs}					%paquete para realización de tablas profesionales

\title{Práctica 4 DSS \\ Manual de instalación}
\author{Alejandro Campoy Nieves \\ Luis Gallego Quero}
\date{\today}

\begin{document}
\maketitle			
 
\section{Web APP}

Para el correcto funcionamiento del proyecto que presentamos, en la parte del servicio web habrá que realizar los siguientes pasos a la hora de la instalación. \\

Como primer paso, importar la base de datos consorcio.sql a mysql, la cual podemos encontrarla en el directorio principal de la carpeta DSS-P4. Posteriormente podemos añadir la carpeta DSS-P4 a nuestra carpeta de eclipse y abrir el proyecto en dicha herramienta. \\

Un paso importante es comprobar que todas las librerías que podemos encontrar en DSS-P4 $>$ WebContent $>$ lib están correctamente insertadas en el Build Path. Una vez llegados a este punto ya estamos listos para hacer Run Server del proyecto. \\

Como último detalle en el paquete servidor, archivo Db.java se ha establecido la conexión con la base de datos mysql, siendo el usuario root y no estableciendo contraseña ninguna. Esto hay que cambiarlo en el caso de que los datos de acceso a mysql propio difieran de estos. Sin más, con los pasos presentados el proyecto debería ser completamente funcional. \\

Por último, una vez lanzado el proyecto, al primera pantalla que nos encontramos es la de login, en la cual tenemos dos opciones. O bien acceder con usuario admin y contraseña admin o bien registrar nuestro propio usuario con el cual accederemos directamente a la aplicación. En esta, en la barra de navegación superior podemos encontrar las diferentes secciones que contiene el proyecto.

\section{Android APP}


\end{document}       
%---------------------------------------------------
