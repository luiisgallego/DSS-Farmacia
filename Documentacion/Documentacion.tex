\documentclass[a4paper,12pt,oneside]{article}

\usepackage[spanish, es-tabla]{babel}
\usepackage[hidelinks]{hyperref}
\hypersetup{
    colorlinks=true,
    linkcolor=blue,
    filecolor=magenta,      
    urlcolor=cyan,
}
\usepackage{graphicx}
\usepackage{amssymb, amsmath, amsbsy}
\usepackage{mathptmx}	
\usepackage{float}
\usepackage{booktabs}					%paquete para realización de tablas profesionales

\title{Práctica 4 DSS \\ Manual de instalación}
\author{Alejandro Campoy Nieves \\ Luis Gallego Quero}
\date{\today}

\begin{document}
\maketitle			
 
\section{Web APP}

Para el correcto funcionamiento del proyecto que presentamos, en la parte del servicio web habrá que realizar los siguientes pasos a la hora de la instalación. \\

Como primer paso, importar la base de datos consorcio.sql a mysql, la cual podemos encontrarla en el directorio principal de la carpeta DSS-P4. Posteriormente podemos añadir la carpeta DSS-P4 a nuestra carpeta de eclipse y abrir el proyecto en dicha herramienta. \\

Un paso importante es comprobar que todas las librerías que podemos encontrar en DSS-P4 $>$ WebContent $>$ lib están correctamente insertadas en el Build Path. Una vez llegados a este punto ya estamos listos para hacer Run Server del proyecto. \\

Como último detalle en el paquete servidor, archivo Db.java se ha establecido la conexión con la base de datos mysql, siendo el usuario root y no estableciendo contraseña ninguna. Esto hay que cambiarlo en el caso de que los datos de acceso a mysql propio difieran de estos. Sin más, con los pasos presentados el proyecto debería ser completamente funcional. \\

Por último, una vez lanzado el proyecto, al primera pantalla que nos encontramos es la de login, en la cual tenemos dos opciones. O bien acceder con usuario admin y contraseña admin o bien registrar nuestro propio usuario con el cual accederemos directamente a la aplicación. En esta, en la barra de navegación superior podemos encontrar las diferentes secciones que contiene el proyecto.

\section{Android APP}

Para el correcto funcionamiento del proyecto que presentamos, en la parte de Android habrá que realizar los siguientes pasos a la hora de utilizarlo.\\

Primero, es importante que el servidor web este operativo siguiendo las instrucciones anteriormente descritas, junto con la base de datos de consorcio en mysql.\\

Es importante destacar que se ha conectado Android a este servicio por medio de una red privada, esto quiere decir que las URLs utilizadas por Android para realizar las distintas peticiones REST no utilizan ``localhost'', si no 192.168.0.158 (IP privada del ordenador con el servicio en ese momento). Si se cambia esta ip deberá de ser modificada en el código de Android.\\

Otro punto importante, si la conexión en las peticiones nos aparece un Time Out (no un Connection Failure), seguramente sea debido a el Firewall, se puede solucionar desactivándolo temporalmente.\\

Para probar el proyecto simplemente basta con abrir Android Studio, importar un nuevo proyecto y la carpeta ``Android'' de esta práctica debería ser reconocida por el IDE. \\

Finalmente para ejecutarlo nos vamos a la pestaña de ``Run $>$ Run app'' y Android Studio debe reconocer el dispositivo móvil. \\

Si no aparece el dispositivo comprobar:

\begin{enumerate}
	\item El dispositivo se encuentra conectado por USB al PC.
	\item Habilitar las \href{https://www.xatakandroid.com/sistema-operativo/opciones-de-desarrollo-de-android-para-que-sirven-y-cuales-deberiamos-activar}{opciones de desarrollador} en el dispositivo.
	\item Permitir \href{https://androidstudiofaqs.com/tutoriales/android-studio-no-reconoce-movil-solucion}{depuración por USB} en el dispositivo.
\end{enumerate}

Se implementa una base de datos interna SQLite a parte del servicio web con las base de datos en mysql para el consorcio. La finalidad de ésta es facilitar tareas tales como el log in de un nuevo usuario y la compra de productos (los pedidos se hacen mediante POST al servicio web igualmente). \\

Las pruebas de esta práctica se han realizado con un Samsung Galaxy S7 Edge con una versión de Android 7.0, aunque debería de ser compatible con cualquier versión.


\end{document}       
%---------------------------------------------------
